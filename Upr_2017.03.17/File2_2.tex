\documentclass{article}
\usepackage{amsmath}
\usepackage{graphicx}
\title{Why I am a genius}

\author{Kanye West}

\date{\today}

\begin{document}

\maketitle

\begin{abstract}

Text

\section{Introduction}
\label{sec:intro}
In Section~\ref{sec:method}, we \ldots

\begin{figure}[htbp] %here->top->bottom->page
\centering
\includegraphics[
width=0.4\textwidth,
angle=-45]{chick}
\caption{Text\label{fig:1}}
\end{figure}

\includegraphics[
width=0.1\textwidth,
angle=45]{chick}

\subsection{Method}

\label{sec:method}

Text %\ldots

\begin{equation}
\label{eq:euler}
e^{i/pi} + 1 = 0
\end{equation}

\emph{By~\eqref{eq:euler}, we have} \ldots

\section{Chapter 2}

\begin{tabular}{lrr} %left->right->right
%������������ �� ������ �� ������
Item & Qty & Unit \$ \\
Widget & 1 & 199.99 \\
Gadget & 2 & 399.99 \\
Cable & 3 & 19.99 \\
\end{tabular}
\\
\\
\\
\begin{tabular}{|l|r|r|} \hline
Item & Qty & Unit \$ \\\hline
Widget & 1 & 199.99 \\
Gadget & 2 & 399.99 \\
Cable & 3 & 19.99 \\\hline
\end{tabular}
\\
\\
\\
\begin{tabular}{lrr} %\hline
Item & Qty & Unit \$ \\%\hline
Widget & 1 & 199.99 \\\hline
Gadget & 2 & 399.99 \\\hline
Cable & 3 & 19.99 \\%\hline
\end{tabular}
\\
\\
\\
\begin{tabular}{l|r|r}
X & O & O  \\\hline
X & X & O  \\\hline
X & O & X  
\end{tabular}
\\
\\
\\
\begin{table}
\centering %{\centering . . .}

\end{table}

\subsection{Section}

Text. %\ldots

\subsection{Section}

\subsection{Section}

\section{Chapter 3}

\section{Chapter 4}

\end{abstract}

\end{document}
