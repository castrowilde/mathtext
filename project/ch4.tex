\chapter[Мерки. Теореми на Жордан-Хан, Лебег и Радон-Никодим]
{Мерки. Теореми на Жордан-Хан,\\ Лебег и Радон-Никодим}\label{ch:4}

С понятието "вероятностна мярка"\, читателят е запознат от изложението в глава 1.
Отличителни черти на вероятностните мерки са : а) те са нормирани ($\bbP (\gO)=1$)
и б) те са неотрицателни ($\bbP(A)\geq 0$). Ще напомним, че за вероятностните
мерки свойствата адитивност и непрекъснатост в нулата са еквивалентни на изброима
адитивност. Именно последното свойство лежи в основата на общото понятие "мярка".
\begin{dfn}\label{dfn:4.1}
Мярка $\mu$ в измеримото пространство $(\gO,\calF)$ се
нарича всяка функция $\mu = \mu(A)$, дефинирана в $\gs$-алгебрата $\calF$ и
приемаща стойности в $(-\infty,+\infty]$, която удовлетворява условията:
\begin{enumerate}
\item $\mu(\es) = 0$
\item $\mu(\sum\limits_{j \in J}A_{j}) = \sum\limits_{j\in J}\mu(A_j)$
\end{enumerate}
за всяко изброимо семейство $\{A_{j},j \in J\}\subseteq \calF$ от непресичащи се
$(A_{i} \cap A_{j} = \es, i \neq j)$ подмножества на $\gO$. Мярката $\mu$ се 
нарича положителна $(\mu\geq0)$, ако $\mu(A)\geq 0,
A\in\calF$, и ограничена, ако $\sup\limits_{A \in \calF}|\mu(A)|<\infty$. Всяка
положителна и нормирана $(\mu(\gO) = 1)$, а следователно и ограничена, мярка
$\mu$ е вероятностна мярка в $(\gO,\calF)$.
\end{dfn}
В тази глава ще изложим три класически теореми, разкриващи структурата на всяка
мярка и съотношенията (абсолютна непрекъснатост, сингулярност) между мерките,
представляващи най-голям интерес за теорията на вероятностите.
\begin{thm}\label{thm:4.1} \textbf{(Жордан-Хан)}. Нека $\mu$ е мярка в
$(\gO,\calF)$. Тогава
\begin{enumerate}
\item Равенствата
\begin{align}
\mu^{+}(A) = \sup\{\mu(B),B\subseteq A\}\notag\\
\mu^{-}(A) = \sup\{-\mu(B),B\subseteq A\}\notag
\end{align}
дефинират две положителни мерки $\mu^{+}$ и $\mu^{-}$ в $(\gO,\calF)$.
\item Мярката $\mu^{-}$ е ограничена и $\mu = \mu^{+} - \mu^{-}$.
\item Съществува множество $D \in \calF$ такова, че
\begin{ena}
\item $\mu(A) \geq 0$ за всяко $A \subseteq D$,
\item $\mu(A) \leq 0$ за всяко $A \subseteq \bar{D}$,
\end{ena}
и следователно $\mu^{+}(A) = \mu(A \cap D), \mu^{-}(A) = - \mu(A \cap
\bar{D}), A \in \calF$.
\end{enumerate}
\end{thm}
\begin{proof}[\textbf{Доказателство}]
Доказателството ще разделим на няколко етапа с цел отделните моменти да бъдат
подробно изяснени. Ключов момент в него е твърдението 3), свързано със съществуването
на подходящо множество $D \in \calF$.
\begin{enumerate}[\indent A.]
\item\label{4.A}
Дефинираме класа $\calB = \{B \in \calF, \mu^{+}(B)=0\}$. Този клас $\calB$
е затворен относно изброими обединения. Наистина, ако $A \subseteq \bigcup\limits
_{n \geq 0} B_{n}, B_{n} \in \calB, n \geq 0$, то
\[
\mu(A) = \sum\limits_{n \geq 0}\mu(A \cap (B_{n}\backslash\bigcup\limits_{m<n}
B_{m})) \leq \sum\limits_{n \geq 0} \mu^{+}(B_{n})=0.
\]
Следователно, $\bigcup\limits_{n \geq 0}B_{n} \in  \calB$.
\item\label{4.B}
Точната долна граница $\beta=\inf\{\mu(B),B \in \calB\}$ \textbf{се достига} за
някое множество от $\calB \subseteq \calF$ и следователно $-\infty < \beta \leq
0$. Наистина, ако $B_{n} \in \calB, n \geq 0$, имат свойството $\mu(B_{n})
\xrightarrow[n \rightarrow \infty]{}\beta$, то $\bigcup\limits_{n \geq
0}B_{n} \in \calB$ и за всяко $m \geq 0$ имаме ...
\end{enumerate}
\end{proof}
