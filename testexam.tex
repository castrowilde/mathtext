\documentclass[leqno,11pt]{article}
\usepackage{amsmath}
\usepackage{amsfonts}
\usepackage{cmap} % cmap позволява търсене на думи на кирилица
\usepackage[T2A]{fontenc} % кодировка за шрифтове на кирилица
\usepackage[cp1251]{inputenc} % кодировка windows-1251 за кирилица
\usepackage[english,bulgarian]{babel} % писане на български и английски език, последният език е текущ
%\usepackage{bibtex}
\newtheorem{theorem}{Теорема}
\let\partial\relax
\DeclareMathOperator{\partial}{d}
\let\i\relax
\DeclareMathOperator{\i}{i}
\let\exp\relax
\DeclareMathOperator{\exp}{e}
\title{Изпит}

\author{Християн Емилов Марков, ф.н. 00147}
\date{24 май 2017 г.}
\begin{document}

\maketitle

\tableofcontents
\section*{Увод}
Примерите за този изпит са взети от \cite{sitectan}.
\section{Производни}
\textbf{Правило за диференциране:}
$\{f[g[h(x)]]\}' = f'[g[h(x)]]g'[h(x)]h'(x)$.
Производната от \textit{n}-ти ред на произведението на функциите $f(x)$ и $g(x)$ се пресмята по формулата:
\begin{equation}\label{eq:one}
\frac{\partial^n}{\partial x^n}[f(x)g(x)] = \sum\limits_{i=0}^{n}\binom{n}{i}f^{(i)}(x)g^{(n-i)}(x).
\end{equation}
\section{Комплексен анализ}
\begin{theorem}[за резидуумите]\label{th:one}
Нека $f$ e аналитична функция в областта $G$ с изключение на изолираните особени точки $a_1, a_2, \dots, a_m$. Ако $\gamma$ е затворена ректифицируема крива в $G$, която не минава през нито една от точките $a_k$, и ако $\gamma \approx 0$ в $G$, то
\end{theorem}
\begin{equation}
\frac{1}{2\pi \i} \int\limits_{\gamma}f = \sum\limits_{k=1}^{m}n(\gamma;a_{k})Res(f;a_k).
\end{equation}
\begin{theorem}[за максимума]\label{th:two}
Нека $G$ е ограничено отворено множество в $\mathbb{C}$ и $f$ е непрекъсната функция в $G^{-}$, която е аналитична в $G$. Тогава
\end{theorem}
\begin{equation}\notag
\text{max}{|f(z)|: z \in G^{-}} = \text{max}\{|f(z)|: z \in \partial G\}.
\end{equation}

\section{Гама функция}
\textit{Гама функцията $\Gamma(z)$ се дефинира като}

\begin{equation} \label{eq:three}
\Gamma(z)\equiv \lim\limits_{n \to \infty} \frac{(n+1)^z n!}{z(z+1) \cdots (z+n)} \equiv \int_{0}^{\infty} \exp^{-t}t^{z-1} \, \partial t.
\end{equation}
От (\ref{eq:three}) се вижда, че $\Gamma(1)=1$. Може да се докаже, че за цели положителни числа $n$

\begin{equation}\notag
\begin{split}
\Gamma(n) &= (n-1)\Gamma(n-1) = (n-1)(n-2)\Gamma(n-2) = \dots\\
&= (n-1)(n-2) \dots 1 = (n-1)!.
\end{split}
\end{equation}

\section*{Заключение}
Теореми \ref{th:one} и \ref{th:two} се изучават във всеки курс по комплексен анализ, а формула (\ref{eq:one}) би трябвало да е известна на всеки студент по математика.

\begin{thebibliography}{1}
\bibitem{sitectan}{\verb'www.ctan.org'}
\end{thebibliography}
\end{document} 
